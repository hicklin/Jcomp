\documentclass[twoside,a4paper,floatfix,twocolumn,superscriptaddress]{article}%twoside
\usepackage{../packages/xjenza}
\usepackage{datetime,tabularx,multirow,subfigure,dblfloatfix}
\usepackage{graphicx,float}
\usepackage{fancyhdr,calc}
\usepackage[hidelinks]{hyperref}
\usepackage{lipsum,cuted,setspace,enumerate}
\usepackage[ansinew,utf8]{inputenc}
\usepackage[compact]{titlesec}
\usepackage{changepage}
\usepackage{xcolor,colortbl,color}
\usepackage{soul,hyphenat}
\usepackage{algpseudocode}
\usepackage{algorithm}
\usepackage{textgreek, upgreek, tipa}% these should be before amssymb, amsthm and amsmath
\usepackage{amssymb,amsthm}
\usepackage{relsize}
\usepackage{epstopdf}
\usepackage{siunitx}
\usepackage{mhchem}
\usepackage{booktabs}
\usepackage{balance}
\usepackage{enumitem}
\usepackage{placeins}
\setlist{noitemsep}
\usepackage{latexsym}
\usepackage{pdfpages}

%--Item spacing: add \begin{enumerate}[noitemsep,topsep=0pt,
%--parsep=0pt,partopsep=0pt] to use--:::::::::::::::::::::::::::::::::::::::::::::

\usepackage{enumitem}
\setitemize{noitemsep,topsep=0pt,parsep=0pt,partopsep=0pt}
%---

%--paragraph spacing--::::::::::::::::::::::::::::::::::::::::::::::::::::::::::::
%\parindent 10 pt
%\parskip 0 pt
%---

%::::::::::::::::::::::::--End of xjenza-preamble.tex--::::::::::::::::::::::::::%

\newcounter{pagna}
\pagenumbering{roman}
\setcounter{pagna}{2}
\setcounter{page}{2}

\begin{document}

\fmstyle
\footnotesize

\section*{Scope of Journal}

Xjenza is the Journal of the Malta Chamber of Scientists and is published in an electronic format. Xjenza is a peer-reviewed, open access international journal. The scope
of the journal encompasses research articles, original research reports, reviews, short communications and scientific commentaries in the fields of: mathematics, statistics, geology, engineering, computer science, social sciences, natural and earth sciences, technological
sciences, linguistics, industrial, nanotechnology, biology, chemistry, physics, zoology, medical studies, electronics and all other applied and theoretical aspect of science.

The first issue of the journal was published in 1996 and the last (No. 12) in 2007. The new editorial board has been formed with internationally recognised scientists, we are planning to restart publication of Xjenza, with two issues being produced every year. One of the aims of Xjenza, besides highlighting the exciting research being performed nationally and internationally by Maltese scholars, is to provide insight to a wide scope of potential authors, including students and young researchers, into scientific publishing in a peer-reviewed environment.

\section*{Instructions for Authors}

Xjenza is the journal of the Malta Chamber of Scientists and is published by the Chamber in electronic format on the website: \href{http://www.mcs.org.mt/index.php/xjenza}{http://www.mcs.org.mt/index.php/xjenza}. Xjenza will consider manuscripts for publication on a wide variety of scientific topics in the following categories

\begin{enumerate}
\item Communications
\item Research Articles
\item Research Reports
\item Reviews
\item Notes
\item News
\item Autobiography
\end{enumerate}

\paragraph{Communications} are short peer-reviewed research articles (limited to three journal pages) that describe new important results meriting urgent publication. These are often followed by a full Research Article.

\paragraph{Research Articles} form the main category of scientific papers submitted to Xjenza. The same standards of scientific content and quality that applies to Communications also apply to Research Articles.

\paragraph{Research Reports} are extended reports describing research of interest to a wide scientific audience characteristic of Xjenza. Please contact the editor to discuss the suitability of topics for Research Reports.

\paragraph{Review Articles} describe work of interest to the wide readership characteristic of Xjenza. They should provide an in-depth understanding of significant topics in the sciences and a critical discussion of the existing state of knowledge on a topic based on primary literature. Review Articles should not normally exceed 6000 words.Authors are strongly advised to contact the Editorial Board before writing a Review.

\paragraph{Notes} are fully referenced, peer-reviewed short articles limited to three journal pages that describe new theories, concepts and developments made by the authors in any branch of science and technology. Notes need not contain results from experimental or simulation work.

\paragraph{News:} The News section provides a space for articles up to three journal pages in length describing leading developments in any field of science and technology or for reporting items such as conference reports. The Editor reserves the right to modify or reject articles for consideration as `news items'.

\paragraph{Errata:} Xjenza also publishes errata, in which authors correct significant errors of substance in their published manuscripts. The title should read: Erratum: ``Original title" by ***, Xjenza, vol. *** (year). Errata should be short and consistent for clarity.

\paragraph{Invited Articles and Special Issues:} Xjenza regularly publishes Invited Articles and Special Issues that consist of articles written on invitation by the Editor or member of the editorial board.

\section*{Submission of Manuscripts}

Manuscripts should be sent according to the guidelines given hereafter to \href{mailto:submissionxjenzaonline@gmail.com}{submissionxjenzaonline@gmail.com}.

%\noindent Prof. Giuseppe Di Giovanni 
%Department of Physiology and Biochemistry 
%Faculty of Medicine and Surgery (On Campus) 
%University of Malta 
%Msida MSD 06 
%Malta 
%Tel: (+356) 2340 2776 
%e-mail: xjenza@mcs.org.mt 

\paragraph{Referees}

All manuscripts submitted to Xjenza are peer reviewed. Authors are requested to submit with their manuscript the names and addresses of three referees, preferably from overseas. Every effort will be made to use the recommended reviewers; however the editor reserves the right to also consult other competent reviewers.

\paragraph{Conflict of Interest}

Authors are expected to disclose any commercial or other associations that could pose a conflict of interest in connection with the submitted manuscript. All funding sources supporting the work, and institutional or corporate affiliations of the authors, should be acknowledged on the title page or at the end of the article.

\paragraph{Policy and Ethics}

The work described in the submitted manuscript must have been carried out in accordance with The Code of Ethics of the World Medical Association (Declaration of Helsinki) for experiments involving humans (\url{http://www.wma.net/en/30publications/10policies/b3/index.html}); EU Directive 2010/63/EU for animal experiments (\url{http://ec.europa.eu/environment/chemicals/lab\_animals/legislation\_en.htm}); Uniform Requirements for manuscripts submitted to Biomedical journals (\url{http://www.icmje.org}). This must be stated at an appropriate point in the article.

\paragraph{Submission, Declaration and Verification}

Submission of a manuscript implies that the work described has not been published previously (except in the form of an abstract or as part of a published lecture or academic thesis), that it is not under consideration for publication elsewhere, that it has been approved for publication by all authors, and tacitly or explicitly, by the responsible authorities where the work was carried out, and that, if accepted, it will not be published elsewhere in the same form, in English or in any other language, including electronically, without the written consent of the copyright-holder.

\paragraph{Permissions}

It is the responsibility of the corresponding author of a manuscript to ensure that there is no infringement of copyright when submitting material to Xjenza. In particular, when material is copied from other sources, a written statement is required from both the author and/or publisher giving permission for reproduction.
Manuscripts in press, unpublished data and personal communications are discouraged; however, corresponding authors are expected to obtain permission in writing from at least one author of such materials.

\section*{Preparation of Manuscripts}

Xjenza accepts submissions in MS Word, Libre Office Writer and \LaTeX{} with the latter being the preferred option. Anyone submitting in \LaTeX{} should use the journal template, the latest version of which can be found at \url{http://github.com/hicklin/Xjenza-Journal-Template}. All the necessary files to run the \LaTeX{} document should be supplied together with the rendered PDF.

If a word processor is used the styling should be kept to a minimum, only introducing bold face, italics, subscript and superscript text where the context requires it. Text should be in single-column format and the word processor options should not be used in order to justify text or hyphenate words. Together with the native format of the word processor, a pdf, generated by the word processor, must be given. Furthermore, artwork should be in accordance to the artwork guidelines give below and must be submitted separately from the word processor file. Similarly, the bibliographic data of the cited material should be submitted separately as an Endnote (*.xml), Research Information Systems (*.ris), Zotero Library (zotero.splite) or a BiB\TeX{} (*.bib) file.

\subsection*{Article Structure}

A manuscript for publication in Xjenza will ordinarily consist of the following: Title page with contact information, Abstract, Highlights, Keywords, Abbreviations, Introduction, Materials and Methods, Results, Discussion, Conclusion, Appendices and References.

The manuscript will be divided into clearly defined numbered sections. Each numbered subsection should be given a brief heading. Each heading should appear on its own separate line. Subsections should be used as much as possible when cross-referencing text: refer to the subsection by the section number as opposed to simply `the text'.

\paragraph{Title page}
\begin{itemize}
\item Title should be concise yet informative. Titles are often used in information-retrieval systems. Avoid abbreviations and formulae where possible.
\item Author names and affiliations. Present the authors' affiliation addresses (where the actual work was done) below the names. Indicate all affiliations with a lower-case superscript number immediately after each author's name and in front of the appropriate address. Provide full postal address of each affiliation, including the country name and, if available, the e-mail address.
\item Corresponding author. Clearly indicate who will handle correspondence at all stages of refereeing and publication, including post-publication. Ensure that telephone and fax numbers (with country and area code) are provided in addition to the e-mail address and complete postal address. Contact details must be kept up to date by the corresponding author.
\item Present/permanent address. If an author has changed the address since the work described, this can be indicated as a footnote to the author's name. The address at which the author actually did the work must be retained as the main, affiliation address. Superscript Arabic numerals are used for such footnotes.
\end{itemize}

\paragraph{Abstract}

A concise and factual abstract is required of up to about 250 words. The abstract should state briefly the background and purpose of the research, the principal results and major conclusions. An abstract is often presented separately from the article, so it must be able to stand alone. For this reason, references and non-standard abbreviations should be avoided. If essential, these must be defined  at first mention in the abstract itself.
%
%bparagraph{Highlights}
%
%Highlights are mandatory for Xjenza. They consist of a short collection of 3-5 bullet points of a minimum of 85 characters (including spaces) each, that convey the core findings of the article and should be submitted in a separate file. Please use `Highlights' in the file name.
%
%bparagraph{Keywords}
%
%Immediately after the abstract, provide a maximum of 10 keywords to be used for indexing purposes.

\paragraph{Abbreviations}

Define abbreviations that are not standard in this field in a footnote to be placed on the first page of the article. Such abbreviations that are unavoidable in the abstract must be defined at their first mention as well as in the footnote and should be used consistenly throughout the text.

\paragraph{Introduction}

State the objectives of the work and provide an adequate background, avoid a detailed literature survey or a summary of the results.

\paragraph{Material and Methods}

Provide sufficient detail to allow the work to be reproduced. Methods already published should be indicated by a reference: only relevant modifications should be described.

\paragraph{Results}

Results should be clear and concise. Numbered/tabulated information and/or figures should also be included.

\paragraph{Discussion}

This should explore the significance of the results of the work, yet not repeat them. Avoid extensive citations and discussion of published literature. A combined section of Results and Discussion is often appropriate.

\paragraph{Conclusion}

The main conclusions based on results of the study may be presented in a short Conclusions section. This may stand alone or form a subsection of a Discussion or Results and Discussion section.

\paragraph{Appendices}

Formulae and equations in appendices should be given separate numbering: Eq. (A.1), Eq. (A.2), etc.; in a subsequent appendix, Eq. (B.1) and so on. Similarly for tables and figures: Table A.1; Fig. A.1, etc.

\paragraph{Acknowledgements}

Collate acknowledgements in a separate section at the end of the article before the references and do not, therefore, include them on the title page, as a footnote to the title or otherwise. List here those individuals who provided assistance during the research (e.g., providing language help, writing assistance or proof reading the article, etc.).

\paragraph{Units}

Follow internationally accepted rules and conventions: use the international system of units (SI). If other units are mentioned, please give their equivalent in SI. Anyone using \LaTeX{} should use the package \href{https://www.ctan.org/pkg/siunitx?lang=en}{siunitx} in all cases.

\paragraph{Footnotes}

Footnotes should be used sparingly. Number them consecutively throughout the article, using superscript Arabic numbers. Many word processors build footnotes into the text, and this feature may be used. Should this not be the case, indicate the position of footnotes by a superscript number in the text and present the footnotes themselves separately at the end of the article. Do not include footnotes in the Reference list.

\paragraph{Table Footnotes}

Indicate each footnote in a table with a superscript lower case letter.

\paragraph{Artwork}

Electronic artwork
General points:
\begin{itemize}
\item Make sure you use uniform lettering and sizing of your original artwork.
\item Save text in illustrations as `graphics' or enclose the font.
\item Only use the following fonts in your illustrations: Arial, Courier, Times, Symbol or Computer Modern Roman, the latter is preferred.
\item Number the illustrations according to their sequence in the text.
\item Name your artwork files as `fig$x$' or `tab$x$' where $x$ corresponds to the sequence number in your document.%Use a logical naming convention for your artwork files.
\item Provide captions to illustrations separately.
\item Produce images near to the desired size of the printed version or grater.
\item Make sure that the artwork has no margins and borders.
\item Submit each figure as a separate file.
\end{itemize}

%A detailed guide on electronic artwork is available on our website: \url{http://www.mcs.org.mt/index.php/xjenza/submission-guidelines}

\paragraph{Formats}

Regardless of the application used, when your electronic artwork is finalised its file format should be one of the following (note the resolution requirements for line drawings, halftones, and line/halftone combinations given below):

\begin{itemize}
\item PDF or SVG: Vector drawings. Embed the font or save the text as `graphics'.
\item JPEG or PNG: Color or grayscale photographs (halftones): always use a minimum of 300 dpi.
\item JPEG or PNG: Bitmapped line drawings: use a minimum of 1000 dpi.
\item JPEG or PNG: Combinations bitmapped line/half-tone (color or grayscale): a minimum of 500 dpi is required.
\end{itemize}

Where possible use a vector format for your artwork (PDF or SVG). If this is not possible, supply files that have and adequate resolution.

\paragraph{Colour Artwork}

Make sure that color artwork files are in an acceptable format (JPEG, PNG, PDF or SVG) and have the correct resolution.

\paragraph{Figure Captions}

Ensure that each illustration has a caption. Supply captions separately, not attached to the figure. A caption should comprise a brief title (not on the figure itself) and a description of the illustration. Keep text in the illustrations themselves to a minimum, but explain all symbols and abbreviations used.

\paragraph{Tables}

Number tables consecutively in accordance with their appearance in the text. Place footnotes to tables below the table body and indicate them with superscript lowercase letters. Avoid vertical rules. Be sparing in the use of tables and ensure that the data presented in tables do not duplicate results described elsewhere in the article. Large tables should be submitted in CSV format.

\paragraph{Citations and References}

Reference and citation styles for manuscripts submitted to Xjenza should be in accordance to the \href{http://www.apastyle.org/}{APA v6} style.


\subparagraph{Citation in text}

References to cited literature in the text should be given in the form of an author's surname and the year of publication of the paper with the addition of a letter for references to several publications of the author in the same year. For further information regarding multiple authors consult the \href{http://www.apastyle.org/}{APA v6} guidelines. Citations may be made directly

\vspace{0.5ex}
Kramer et al. (2010) have recently shown \ldots{}

\vspace{0.5ex}
\noindent{}or parenthetically

\vspace{0.5ex}
as demonstrated (Allan, 2000a, 2000b, 1999; Allan and Jones, 1999).

\noindent{}Groups of references should be listed first alphabetically, then chronologically. When writing in \LaTeX{} use \verb|\textcite{}| and \verb|\parencite{}| for the respective cases mentioned.

\subparagraph{The reference section}

Every reference cited in the text should also be present in the reference list (and vice versa). The reference list should also be supplied as an Endnote (*.xml), Research Information Systems (*.ris), Zotero Library (zotero.splite) or a BiB\TeX{} (*.bib) file. Unpublished results and personal communications are not recommended in the reference list, but may be mentioned in the text. If these references are included in the reference list they should follow the standard reference style of the journal and should include a substitution of the publication date with either `Unpublished results' or `Personal communication'. Citation of a reference as `in press' implies that the item has been accepted for publication.

References should be arranged first alphabetically and then further sorted chronologically if necessary. More than one reference from the same author(s) in the same year must be identified by the letters 'a', 'b', 'c', etc., placed after the year of publication. Consult the \href{http://www.apastyle.org/}{APA v6} guidelines for multiple authors. Below are some examples of referencing different bibliographic material.

\subparagraph{Reference to a Journal Publication:}
\begin{description}
\item[] \hangindent=2em Agree, E.~M. and Freedman, V.~A. (2011). \newblock {A Quality-of-Life Scale for Assistive Technology: Results of a Pilot Study of Aging and Technology}. \newblock {\em Phys. Ther.}, 91(12):1780--1788.
\item[] \hangindent=2em McCreadie, C. and Tinker, A. (2005). \newblock {The acceptability of assistive technology to older people}. \newblock {\em Ageing Soc.}, 25(1):91--110.
\end{description}

\subparagraph{Reference to a Book:}
\begin{description}
\item[] \hangindent=2em Brownsell, B. (2003). \newblock {\em {Assistive Technology and Telecare: Forging Solutions for Independent Living}}.\newblock Policy Press, Bristol.
\item[] \hangindent=2em Fisk, M.~J. (2003). \newblock {\em {Social Alarms to Telecare: Older People's Services in Transition}}.\newblock Policy Press, Bristol, 1st edition.
\end{description}

\subparagraph{Reference to a Chapter in an Edited Book:}
\begin{description}
\item[] \hangindent=2em Brownsell, S. and Bradley, D. (2003). \newblock {New Generations of Telecare Equipment}. \newblock In {\em Assist. Technol. Telecare Forg. Solut. Indep. Living}, pages 39--50.
\end{description}

\subparagraph{Web references}

The full URL should be given together with the date the reference was last accessed. Any further information, if known (DOI, author names, dates, reference to a source publication, etc.), should also be given. Web references can be listed separately or can be included in the reference list.

\subparagraph{References in a Special Issue}

Please ensure that the words `this issue' are added to any references in the list (and any citations in the text) to other articles in the same Special Issue.

\subparagraph{Journal Abbreviations}
Journal names should be abbreviated according to:

-Index Medicus journal abbreviations: \url{http://www.nlm.nih.gov/tsd/serials/lji.html};

-List of title word abbreviations: \url{http://www.issn.org/2-22661-LTWA-online.php};

-CAS (Chemical Abstracts Service): \url{http://www.cas.org/sent.html}.

\paragraph{Video data}

Xjenza accepts video material and animation sequences to support and enhance the presentation of the scientific research. Authors who have video or animation files that they wish to submit with their article should send them as a separate file. Reference to the video material should be clearly made in text. This will the modified into a linked to the paper's supplementary information page. All submitted files should be properly labelled so that they directly relate to the video files content. This should be within a maximum size of 50 MB.

\section*{Submission check list}

The following list will be useful during the final checking of a manuscript prior to sending it to the journal for review. Please consult this Guide for Authors for further details of any item.

\vspace{0.5ex}
\begin{itemize}
\item One author has been designated as the corresponding author with contact details:
\begin{itemize}
\item E-mail address.
\item Full postal address.
\item Telephone and fax numbers.
\end{itemize}

\item All necessary files have been sent, and contain:
\begin{itemize}
%\item Keywords
\item All figures are given separately in PDF, SVG, JPEG of PNG format.
\item Caption for figures is included at the end of the text.
\item All tables (including title, description, footnotes) are included in the text and large tables have been given separately as CSV.
\item The reference list has been given in XML, RIS, zotero.splite or BIB file format.
\end{itemize}
%\balance

\item Further considerations
\begin{itemize}
\item Abstract does not exceed about 250 words.
\item Manuscript has been `spell-checked' and `grammar-checked'.
\item References are in the required format.
\item All references mentioned in the reference list are cited in the text, and vice versa.
\item Bibliographic data for all cited material has been given.
\item Permission has been obtained for use of copyrighted material from other sources (including the Web).
\item A PDF document generated from the word processor used is given.
\end{itemize}
\end{itemize}


\section*{After Acceptance}

\paragraph{Use of the Digital Object Identifier}

The Digital Object Identifier (DOI) may be used to cite and link to electronic documents. The DOI consists of a unique alpha-numeric character string which is assigned to a document by the publisher upon the initial electronic publication. The assigned DOI never changes. Therefore, it is an ideal medium for citing a document, particularly `Articles in press' because they have not yet received their full bibliographic information. When you use a DOI to create links to documents on the web, the DOIs are guaranteed never to change.

\balance
\paragraph{Proofs, Reprints and Copyright}

Authors will normally be sent page proofs by e-mail or fax where available. A list of any necessary corrections should be sent by fax or email to the corresponding editor within a week of proof receipt to avoid unnecessary delays in the publication of the article. Alterations, other than essential corrections to the text of the article, should not be made at this stage. Manuscripts are accepted for publication on the understanding that exclusive copyright is assigned to Xjenza. However, this does not limit the freedom of the author(s) to use material in the articles in any other published works.

%\normalsize
\end{document}